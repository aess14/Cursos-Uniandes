
%--------------------------------------------------------------------
%--------------------------------------------------------------------
% Formato para los talleres del curso de Métodos Computacionales
% Universidad de los Andes
%--------------------------------------------------------------------
%--------------------------------------------------------------------

\documentclass[11pt,letterpaper]{exam}
\usepackage{amsmath}
\usepackage[utf8]{inputenc}
\usepackage[spanish]{babel}
\usepackage{graphicx}
\usepackage{tabularx}
\usepackage[absolute]{textpos} % Para poner una imagen completa en la portada
\usepackage{hyperref}
\usepackage{float}

\newcommand{\base}[1]{\underline{\hspace{#1}}}
\boxedpoints
\pointname{ pt}

\extraheadheight{-0.15in}

\newcommand\upquote[1]{\textquotesingle#1\textquotesingle} % To fix straight quotes in verbatim



\begin{document}
\begin{center}
{\Large Universidad de los Andes - M\'etodos Computacionales Avanzados} \\
Tarea 2 - \textsc{Ecuaciones Diferenciales Parciales}\\
31-03-2017\\
\end{center}

\vspace{0.3cm}


\noindent
La soluci\'on a este ejercicio debe subirse por SICUA antes de las 6:00PM
del viernes 28 de abril del 2017. 
Los c\'odigos deben encontrarse en un unico repositorio de \verb'github'
con el nombre \verb"NombreApellido_Tarea2". Por ejemplo yo deber\'ia
subir crear un repositorio con el nombre \verb"JaimeForero_Tarea2". 

\noindent
En el repositorio deben estar los siguientes elementos.
\begin{itemize}
\item (60 puntos) Un c\'odigo fuente en C que resuelve
  las ecuaciones diferenciales (20 puntos para Shock y 40 puntos para
  Sedov) y produce datos.  
\item (40 puntos) Un c\'odigo en Python que lee los datos producidos por el
  c\'odigo en C y (30 puntos para Shock y 10 puntos para Sedov)
  produce visualizaciones. 
\item (10 puntos) Un makefile que:
\begin{itemize}
\item compila el c\'odigo en C (\verb"make exec").
\item ejecuta el c\'odigo (\verb"make shock", \verb"make sedov").
\item produce graficas en Python (\verb"make plotshock", \verb"make plotsedov").
\end{itemize}
\item (10 puntos) Un archivo de texto (\verb"README") con nombres completos y c\'odigos de los integrantes (m\'inimo dos personas, m\'aximo tres personas).
\end{itemize}
NOTA: Todos los c\'odigos ser\'an ejecutados en una sesi\'on
interactiva de clustergate.

\vspace{0.3cm}

\begin{questions}
\question{\bf{Tubo de Shock}}

Considere un tubo unidimensional de longitud $L=1$ con condiciones
iniciales de densidad $\rho(x\leq 0.5,t=0)=1.0$ y $\rho(x>0.5,t=0)$,
presion $p(x\leq0.5, t=0)=1.0$ y $p(x>0.5,t=0)=0.1$, velocidad
$u(x,0)=0$.

Utilice m\'etodos de diferencias finitas con un esquema de
Lax-Wendroff para resolver las ecuaciones de Euler hasta que la
discontinuidad en la velocidad llegue a $x=0.9$.

El c\'odigo de Python debe graficar el estado final para la densidad,
presi\'on y velocidad junto a la soluci\'on anal\'itica.
 
\question{\bf{Explosi\'on de Sedov}}

Considere un cubo de 256 metros de lado lleno de aire a una
temperatura de $300 K$ y presi\'on atmosf\'erica.
En el centro de este cubo se liberan $10^{10}$ Joules en una regi\'on
c\'ubica de 2 metros de lado. Siga la evoluci\'on de la onda de choque
hasta que su radio sea igual a 120 metros.

Para resolver las ecuaciones de Euler utilice m\'etodos de vol\'umenes
finitos y una malla cartesiana regular con espaciado de $2$ metros en
cada direcci\'on. 

El c\'odigo de Python debe graficar la densidad
promedio en la direcci\'on radial como funci\'on del radio, para
posiciones de la onda de choque en $r=10,60,120$m. La grafica debe
mostrar en el caption el tiempo (en segundos) correspondiente. 

\end{questions}
\end{document}

